%!TEX root = ../template.tex
%%%%%%%%%%%%%%%%%%%%%%%%%%%%%%%%%%%%%%%%%%%%%%%%%%%%%%%%%%%%%%%%%%%%
%% abstract-en.tex
%% NOVA thesis document file
%%
%% Abstract in English([^%]*)
%%%%%%%%%%%%%%%%%%%%%%%%%%%%%%%%%%%%%%%%%%%%%%%%%%%%%%%%%%%%%%%%%%%%

\typeout{NT FILE abstract-en.tex}%

Software verification is a key principle for building and deploying reliable
and correct software. Multiple verification techniques are employed
successfully nowadays, such as theorem proving, model checking, and testing,
each with strengths and trade-offs. These methods help ensure that systems
behave as expected and meet reliability standards.

Runtime Verification is a verification technique that has gained popularity for
its lightweight and complementary approach to ensuring assurance over a system
under scrutiny. Unlike the other traditional methods, runtime verification
leverages the ability to perform checks over the system while it is running,
enabling the dynamic monitoring of a system's behaviour. It adds the capability
to verify and alert the system whenever it begins to behave in an unexpected or
undesirable manner, enhancing the overall reliability and correctness.

In this thesis, we addressed the problem of runtime verification in actor-based
systems, a programming paradigm employed by languages such as \texttt{Erlang}.
We define a specification language, WALTZ, that enables the specification of
properties that span multiple actors and perform causal-aware verification.
After designing and implementing the compiler for WALTZ by generating
\texttt{Erlang} monitor executables, we implemented the orchestration needed to
maintain causal relations between systems that follow the OTP guidelines by
surgically injecting code into the system. With these two components, we built
\texttt{ACTORCHESTRA}, a tool that allows users to specify properties and
automatically monitor them at runtime.

We evaluated two case studies to demonstrate the range of the specification
language and the tool. We also analysed the overhead introduced by the tool
compared to the baseline version of the code and explained the trade-offs of
adding an extra layer of safety to our programs.

% Palavras-chave do resumo em Inglês
% \begin{keywords}
% Keyword 1, Keyword 2, Keyword 3, Keyword 4, Keyword 5, Keyword 6, Keyword 7, Keyword 8, Keyword 9
% \end{keywords}
\keywords{
  runtime verification \and
  monitoring \and
  actor-based programming model \and
  property monitoring \and
  monitorable properties
}
