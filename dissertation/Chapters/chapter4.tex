%!TEX root = ../template.tex
%%%%%%%%%%%%%%%%%%%%%%%%%%%%%%%%%%%%%%%%%%%%%%%%%%%%%%%%%%%%%%%%%%%%
%% chapter4.tex
%% NOVA thesis document file
%%
%% Chapter with lots of dummy text
%%%%%%%%%%%%%%%%%%%%%%%%%%%%%%%%%%%%%%%%%%%%%%%%%%%%%%%%%%%%%%%%%%%%

\typeout{NT FILE chapter4.tex}%

\chapter{Elaboration work plan}
\label{cha:plan}

In this thesis, we propose extending the monitor verdicts beyond simple boolean
outcomes yielded by current monitors, as those give no insight into how the
verdict was produced. In order to address this limitation, we will develop a
blame assignment algorithm to identify the source of the violation in an
actor-based system, offering aid in further debugging.

The steps of the work plan are described below:

\begin{enumerate}
  \item \textbf{Exploring the detectEr tool and its limitations.}
  First, we need to explore the \texttt{detectEr} tool, which uses the security
  fragment of Hennessy-Milner logic (sHML) as its specification language. The
  tool extended the specification language to maxHML$^D$, the most expressive
  fragment of security properties that carry data.

  The tool supports various instrumentation techniques, but some are still
  under development. Additionally, the specification language allows a user to
  define a range of properties in the context of an actor-based model. For
  instance, it is possible to specify the payload of messages exchanged between
  processes.
  \item \textbf{In depth exploration of the capabilities of the tracing mechanism of Erlang.} 
  The \texttt{detectEr} tool uses \texttt{Erlang}'s built-in tracing
  mechanism to analyze systems implemented in \texttt{Erlang/Elixir}. By
  leveraging these native capabilities, we can extract valuable information to
  our advantage and use it as a foundation for developing the blame assignment
  algorithm.

  \item \textbf{Development of the blame assignment algorithm.}
  We propose to enhance the verdicts provided by monitors by developing a blame
  assignment algorithm that identifies the source of violations of specific
  properties in actor-based systems. We aim to integrate the insights gained
  from this algorithm into the verdicts, facilitating debugging.

  Previous work, such as \texttt{WHYMON} and a more recent algorithm for
  actor-based systems~\cite{monitoringLinearTimeExp}, explored how to
  explain the verdicts of a monitor rather than just providing a simple truth
  value indicating whether a given property was violated. However, none of
  these approaches have identified the source of the violation. 

  Building on these foundations, we aim to develop a more sophisticated
  algorithm for blame assignment, enabling the identification of violation
  sources. This approach enhances explainability through fault
  localization and blame assignment, a key area of research in RV~\cite{issues}.

  \item \textbf{Build a case study in Erlang/Elixir.}
  Build a case study, like the one presented on the running example of this
  document, to study the impact and reliability of the blame assignment
  algorithm.

  \item \textbf{Extend the detectEr specification language to MFOTL.}
  \texttt{detectEr} utilizes a fragment of Hennessy-Milner logic. We aim to
  investigate the feasibility of adopting the specification language used in
  the \texttt{WHYMON} tool into an actor-based model. This
  change could enable the expression of richer properties, but we must
  carefully evaluate its impact on monitorability.

  \item \textbf{Exploring partial monitorability.}
  Working with MFOTL raises the problem of non-monitorability. Therefore, we
  will study the applicability of the partial monitorability algorithm defined
  in~\cite{ferrando2025towards} to monitor specific properties that, while
  theoretically non-monitorable, might be worth monitoring in practice.
 
  \item \textbf{Improvement of the detectEr verdicts.}
  With the information gained from the blame assignment algorithm, we propose
  to extend the verdicts of the \texttt{detectEr} tool, enabling an improved
  explainability of such verdicts and easing the debugging process.
 
  \item \textbf{Evaluation.} An evaluation is needed to assess the impact of
  the blame assignment algorithm and various instrumentation techniques in
  actor-based systems within real-world tools implemented in \texttt{Erlang}
  such as \texttt{Cowboy}~\cite{cowboy}.
 
  \item \textbf{Report Writing.}
\end{enumerate}

\Cref{fig:gantt} depicts a Gantt chart for the planned work schedule. An important note, is that the first step was already
performed in this preparation phase, therefore it is not mentioned in the Gantt chart.

\begin{figure}[htbp]
  \centering
  \includegraphics[width=1\linewidth]{Chapters/Figures/ganttfinal.pdf}%
  \caption{Gantt chart of the work plan}
  \label{fig:gantt}
\end{figure}
