\typeout{NT FILE chapter9.tex}%

\chapter{Conclusions}
\label{cha:conclusions}

This chapter finalizes the document with a summary of what this work tackled and providing
future paths to improvement and extensions to the work.

\section{Summary}
\label{sec:summary}

We developed a framework for runtime verification of \texttt{Erlang} systems
that adhere to OTP standards and follow client-server architectures. The effort
involved studying the theoretical foundations of runtime verification,
analysing existing tools, particularly those targeting actor-based systems, and
identifying their capabilities and limitations. Building on this foundation, we
developed three key components: a specification language for defining
properties, a compiler that generates running \texttt{Erlang} monitors from
these specifications, and a causality-tracking entity explicitly designed for
client-server OTP systems.

WALTZ is a specification language designed explicitly for actor-based systems.
It allows users to define properties over chains of interactions between
different processes by capturing the messages deemed relevant. Built with the
concurrency model of \texttt{Erlang} in mind, WALTZ evaluates properties not
only over the observed trace but also within the causal context of each
message. This context-awareness is essential for correctly verifying properties
in concurrent systems, as it prevents the unintended mixing of messages from
different causal chains. One of WALTZ’s key strengths is its ability to specify
properties that span multiple actor interactions, enabling users to define
comprehensive, system-wide specifications or focus on specific interaction
patterns. Equally important is its context-aware semantics, which are critical
for verification in highly concurrent environments.

A central goal of our work was compiling WALTZ properties into running
monitors. The compilation process leverages the close alignment between WALTZ's
semantics and the actor-based model. The generated monitors are scalable,
tracking distinct message chains and evaluating them independently, producing
verdicts upon detecting either a violation or satisfaction. These monitors
operate alongside the system in a structured manner, without interfering with
its execution, acting as oracles that receive only the relevant traces and
evaluate them against the formally defined properties.

The \texttt{conductor} is the entity responsible for tracking all the
interaction in the system, and assigning a causal reference, essentially the
context, to each message that is observed. It is the primary entity that
enables monitors to receive a trace with the context directly attached to the
message payload, thereby avoiding concurrency problems on the monitor's side,
as all important information is sent atomically. The \texttt{conductor} focuses
on client–server applications built with the OTP standard, enabling context
injection by providing a well-defined framework to build upon.

Together, these components form \texttt{ACTORCHESTRA}, a comprehensive runtime
verification framework for verifying context-aware properties in actor-based
systems, enabling the specification and evaluation of properties that span
multiple interacting actors.

\section{Future Work}
\label{sec:future_work}

There are several potential extensions and improvements to our work. For WALTZ,
one such enhancement would be the ability to capture multiple events
co-occurring as part of a property. Currently, the language supports only
ordered chains of messages, but in some scenarios, it may be necessary to
express properties involving concurrent events. Supporting this would require
introducing a new operator to the language to handle these simultaneous
interactions. Additionally, we can introduce logical conjunction and
disjunction operators to increase the expressiveness of the language. Their
inclusion requires the compilation of multiple monitors that coordinate and
interact with one another, such as the handling of nested formulas.

Adding time to properties is also a path to follow, as with time, we will be
able to escape the non-monitorability issue, since we are setting a beginning
and an end to our properties. Another interesting extension would be the
definition of properties that compare or relate two or more different contexts,
essentially allowing inter-context property definitions to enrich our language.
Exploring the notions of unbreakability of chains might also be an interesting
topic, since \texttt{Erlang} is built around a fault-tolerant design, and
sometimes it is normal for things to fail. In our case, we do not permit
failures because of the non-breakable chains of messages. An interesting
direction for future work would be to relax this constraint and define
properties that tolerate failures, provided that the failed event eventually
succeeds.

Regarding the \texttt{conductor} and \texttt{ACTORCHESTRA}, our current
implementation targets a specific system architecture. As a result, certain
types of systems, such as publish-subscribe architectures, do not fit within
our existing assumptions. There is significant potential for improvement by
investigating such systems and developing methods to manage causality
effectively within them. Given the wide variety of system architectures and
behaviours, tracking and handling all of them imposes a great challenge. A
standard reference is necessary, which is why we focused on the specific target
system used in this work. A natural direction for future work is to extend
\texttt{ACTORCHESTRA} to support multiple architectures, enabling it to manage
causality across a broader range of system designs.


There is also room for improvement in the generated monitors, as specific
properties, especially nested properties, could be enhanced to allow for more
parallelism in verification and verdict notification. Not only improvement in
the monitor organisation, but also the entire instrumentation pipeline, in
order to reduce the overhead added to the monitored system.

Finally, given the structure of our specification language and the design of
the generated monitors, extending the framework to include verdict
explainability could significantly enhance the utility of the results. Such an
extension would enable more profound insights into property violations and open
avenues for research into automated blame assignment.




