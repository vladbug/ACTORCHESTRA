%!TEX root = ../template.tex
%%%%%%%%%%%%%%%%%%%%%%%%%%%%%%%%%%%%%%%%%%%%%%%%%%%%%%%%%%%%%%%%%%%%
%% abstract-pt.tex
%% NOVA thesis document file
%%
%% Abstract in Portuguese
%%%%%%%%%%%%%%%%%%%%%%%%%%%%%%%%%%%%%%%%%%%%%%%%%%%%%%%%%%%%%%%%%%%%

\typeout{NT FILE abstract-pt.tex}%


A verificação de software é um princípio fundamental para a construção e
implementação de software fiável e correto. Atualmente, várias técnicas de
verificação são empregues com sucesso: prova de teoremas, verificação de
modelos (model checking) e testes, cada uma com os seus pontos fortes e
compromissos. Estes métodos ajudam a garantir que os sistemas se comportam
conforme o esperado e cumprem os padrões de fiabilidade.

A Verificação em Tempo de Execução (Runtime Verification) é uma técnica de
verificação que se tornou popular e conhecida pela sua abordagem leve e
complementar, oferecendo garantias adicionais sobre um sistema em análise. Ao
contrário dos métodos tradicionais, a verificação em tempo de execução tira
partido da capacidade de realizar verificações enquanto o sistema está em
funcionamento, permitindo a monitorização dinâmica do comportamento do sistema.
Isto acrescenta a possibilidade de verificar e alertar sempre que o sistema
começa a comportar-se de forma inesperada ou indesejada, reforçando assim a
fiabilidade e correção global.

Nesta tese, abordámos o problema da verificação em tempo de execução em
sistemas baseados em atores, um paradigma de programação utilizado por
linguagens como o \texttt{Erlang}. Definimos uma linguagem de especificação,
WALTZ, que permite a definição de propriedades que abrangem múltiplos actores e
realiza uma verificação sensível à causalidade. Após conceber e implementar o
compilador para WALTZ, gerando executáveis de monitores em \texttt{Erlang},
implementámos a orquestração necessária para manter as relações de causalidade
entre sistemas que seguem as diretrizes OTP, através da injeção cirúrgica de
código no sistema. Com estes dois componentes, construímos o
\texttt{ACTORCHESTRA}, a ferramenta que permite ao utilizador especificar
propriedades e monitorizá-las automaticamente em tempo de execução.

Realizámos uma avaliação com base em dois estudos de caso, demonstrando o
alcance da linguagem de especificação e da ferramenta, e analisámos a
sobrecarga causada pela ferramenta em comparação com a versão original do
código, explicando os compromissos associados a ter uma camada adicional de
segurança nos nossos programas.

% Palavras-chave do resumo em Português
% \begin{keywords}
% Palavra-chave 1, Palavra-chave 2, Palavra-chave 3, Palavra-chave 4
% \end{keywords}
\keywords{
  verificação em tempo de execução \and
  monitorização de propriedades \and
  propriedades monitorizávies \and
  modelo de atores
}
% to add an extra black line
